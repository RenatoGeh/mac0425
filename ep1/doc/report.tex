\documentclass[a4paper,10pt]{article}

\usepackage[brazilian]{babel}
\usepackage[utf8]{inputenc}
\usepackage[T1]{fontenc}
\usepackage{titlesec}
\usepackage{graphicx}
\usepackage{mathtools}
\usepackage{amsthm}
\usepackage[top=1.0in,bottom=1.0in]{geometry}
\usepackage{hyperref}
\usepackage[singlelinecheck=false]{caption}
\usepackage{listings}

\newcommand\blfootnote[1]{%
  \begingroup
  \renewcommand\thefootnote{}\footnote{#1}%
  \addtocounter{footnote}{-1}%
  \endgroup
}

\newcommand\defeq{\mathrel{\overset{\makebox[0pt]{\mbox{\normalfont\tiny\sffamily def}}}{=}}}

\titleformat{\section}
  {\normalfont\scshape\bfseries}{\thesection}{1em}{}
\titleformat{\subsection}
  {\normalfont\scshape\bfseries}{\thesubsection}{1em}{}
\titleformat{\paragraph}
  {\normalfont}{\theparagraph}{1em}{}
\titleformat{\subparagraph}
  {\normalfont}{\thesubparagraph}{1em}{}

\captionsetup[table]{labelsep=space}

\theoremstyle{plain}

\newtheorem*{spn-def}{Definição}
\newtheorem*{spn-thm}{Teorema}

\title{\textbf{Relatório do EP1 - MAC0425}}

\begin{document}
\date{}
\author{}
\vspace*{-40pt}
{\let\newpage\relax\maketitle}

Renato Lui Geh

NUSP: 8536030

\section{Resultados}

\paragraph{
  As variáveis na parte superior da tabela são as mesmas que aparecem
na tela de estatísticas do jogador automático do EP no modo teste.
}

\begin{table}[h]
\begin{center}
\captionsetup{justification=centering}
\begin{tabular}{*{3}{c|} c}
DFS \\
\hline
Passos & Gerados & Expandidos & Ramificação \\
\hline
36 & 1701   & 1619    & 1.05 \\
36 & 490    & 406     & 1.21 \\
31 & 57693  & 57622   & 1.00 \\
35 & 144    & 57      & 2.53 \\ 
27 & 727255 & 727202  & 1.00 \\
\end{tabular}
\end{center}
\end{table}

\begin{table}[h]
\begin{center}
\begin{tabular}{*{3}{c|} c}
BFS \\
\hline
Passos & Gerados & Expandidos & Ramificação \\
\hline
13 & 1229 & 1096 & 1.12 \\
13 & 1234 & 1116 & 1.11 \\
15 & 1606 & 1471 & 1.09 \\
1  & 6    &  1   & 6.00 \\
17 & 1912 & 1817 & 1.05 \\
\end{tabular}
\end{center}
\end{table}

\begin{table}[h]
\begin{center}
\begin{tabular}{*{3}{c|}c}
BestFS \\
\hline
Passos & Gerados & Expandidos & Ramificação \\
\hline
13 &  1242 & 1058 & 1.17 \\
13 &  1246 & 1123 & 1.11 \\ 
15 &  1615 & 1418 & 1.14 \\ 
1  &  6    & 1    & 6.00 \\
17 &  1906 & 1691 & 1.13 \\
\end{tabular}
\end{center}
\end{table}

\begin{table}[h]
\begin{center}
\begin{tabular}{*{3}{c|}c}
A-star \\
\hline
Passos & Gerados & Expandidos & Ramificação \\
\hline
13 &  691  &  542  &  1.27 \\ 
13 &  798  &  611  &  1.31 \\
15 &  867  &  704  &  1.23 \\
1  &  6    &  1    &  6.00 \\
17 &  1116 &  946  &  1.18 \\
\end{tabular}
\end{center}
\end{table}

\paragraph{
  O algoritmo A-star foi rodado só com a heurística \texttt{manhattanDistanceAdmissible}, que
é uma distância de Manhattan admissível para o problema. No caso descontamos a posição \texttt{y}
da \texttt{manhattanDistance}, já que o usuário não escolhe ir uma distância para baixo de cada
vez.
}

\section{Comparação}

\paragraph{
  Agora que temos todos os dados vamos compará-los. Para calcular a tabela abaixo, pegou-se cada 
instância $i$ de um algoritmo da tabela acima e, para cada outra instância $j$ de qualquer outro 
algoritmo, fizemos $r_{ij} = \frac{\sum x_i}{\sum x_j} = \sum \frac{x_i}{x_j}$, onde $x_i$ é a 
instância $i$ de uma variável $x$. Por exemplo, para compararmos o número de passos da DFS e BFS, 
fazemos $r_{dfs,bfs} = \sum \frac{passos_{dfs}}{passos_{bfs}} = \frac{165}{59} = 2.79$. Para não 
termos de escrever várias tabelas, comparamos um algoritmo com todos e passamos para o próximo. 
Deste modo temos todas as comparações de um jeito mais compacto. É fácil de ver que podemos 
calcular os valores restantes a partir dessa e das tabelas que vimos na seção passada.
}

\paragraph{
  A partir dessa tabela, podemos ver que a DFS faz muitos passos, pode acabar gerando e expandindo
muitos nós e tem uma média de ramificação muito similar a cada iteração. Mas também dá para se
ver que há casos que ela tem uma performance muito melhor que os outros algoritmos, chegando a ter
menos nós gerados e expandidos que o A-star. No entanto, podemos ver que, na média, a DFS é bem
pior que o resto dos algoritmos. Além disso, a DFS é sub-ótima, o que explica o número alto de 
passos para a solução final quando comparada aos outros.
}

\paragraph{
  A BFS, BestFS e A-star tem mesmo número de passos para a solução final. Isso é por que os três
algoritmos são ótimos (se o custo for uniforme). Eles também tem uma média similar de ramificação,
já que a BFS é, por definição, um algoritmo de largura e BestFS e A-star podem voltar atrás e
partir de um caminho diferente da qual estava seguindo. Apesar de terem resultados um tanto
parecidos, a BFS e BestFS geram e expandem mais em média do que o A-star. De fato, se olharmos para
a tabela desta seção, pode-se ver que a A-star gera e expande mais ou menos metade dos nós que a 
BestFS e BFS, enquanto que estas duas tem média muito parecida. Isso ocorre por que em um espaço
de estados onde o custo é uniforme, a BestFS comporta-se de forma parecida a uma BFS.
}


\begin{table}[h]
\begin{center}
\begin{tabular}{*{5}{c|}c}
Variável & i\textbackslash j &  DFS &   BFS &   BestFS &  A-star \\
\hline
Passos  & DFS  & 1 & 2.79 & 2.79 & 2.79 \\ 
Gerados & BFS  & 0.007 & 1 & 0.99 & 1.72 \\
Expandidos & BestFS & 0.006 & 0.96 & 1 & 1.88 \\
Ramificação & A-Star & 1.61 & 1.05  & 1.04 & 1.05 \\
\end{tabular}
\end{center}
\end{table}

\section{Melhorias}

\paragraph{
  A princípio, quando foram implementadas, todos os algoritmos geravam e expandiam todos os nós que
precisavam sem verificar se aquele nó já tinha sido visto em algum caminho anterior. Ao testar
a BFS, BestFS e A-star, o tempo para calcular cada solução era muito grande, travando o browser.
Para melhorar a performance, foi incluído um \texttt{Set} para tentar guardar os nós vistos e não
entrarmos em um caminho já percorrido. Incluímos esse \texttt{Set} na BFS, BestFS e A-star
para podermos rodar os testes normalmente. Os testes acima foram feitos já com o \texttt{Set} 
incluído.
}

\paragraph{
  A DFS, no entanto, ainda não guardava os nós já visitados. A tabela abaixo mostra a DFS com a
memorização de nós implementada. Dá para ver que a quantidade de nós gerados e expandidos é muito
menor, chegando a ser melhor que o A-star em termos de memória alocada. Contudo, ele ainda é
sub-ótimo.
}

\begin{table}[h]
\begin{center}
\begin{tabular}{*{3}{c|}c}
DFS-stored \\
\hline
Passos & Gerados & Expandidos & Ramificação \\
\hline
36 & 363 & 293 & 1.24 \\
36 & 234 & 162 & 1.44 \\
31 & 698 & 637 & 1.10 \\
35 & 129 & 56 & 2.30 \\
27 & 935 & 889 & 1.05 \\
\end{tabular}
\end{center}
\end{table}

\paragraph{
  Portanto, um jeito de se melhorar os algoritmos é memorizar os nós já vistos para não termos de
percorrer um caminho já percorrido.
}

\paragraph{
  Uma outra possível melhoria é criar heurísticas melhores. A heurística usada pela BestFS é apenas
o custo acumulado de cada nó. O A-star nas tabelas acima usa uma heurística que combina o custo 
acumulado de cada nó e a \texttt{manhattanDistanceAdmissible}, que é apenas o módulo da diferença
das posições no eixo horizontal entre o tetraminó do nó atual com o tetraminó meta.
}

\paragraph{
  Para melhorarmos nossa heurística do A-star, podemos, além da distância de Manhattan admissível,
incluir o quão distante da rotação final está o tetraminó atual. Portanto, a heurística resultante
é:
}

\lstset{language=Java}
\begin{lstlisting}[frame=single]
var betterHeuristics = function(s1, s2) {
  return Math.abs(s1.tetromino.xpos - s2.tetromino.xpos) +
    Math.abs(s1.tetromino.type.next - s2.tetromino.type.next);
}
\end{lstlisting}

\begin{table}[h]
\begin{center}
\begin{tabular}{*{3}{c|}c}
A-star-hr \\
\hline
Passos & Gerados & Expandidos & Ramificação \\
13 & 514 & 375 & 1.37 \\
13 & 596 & 421 & 1.42 \\
15 & 648 & 488 & 1.33 \\
1 & 6 & 1 & 6.00 \\
17 & 891 & 709 & 1.26 \\
\end{tabular}
\end{center}
\end{table}

\paragraph{
  Pela tabela \texttt{A-star-hr}, pode-se ver que o número de nós gerados e expandidos é menor que
o do A-star original. Além disso, \texttt{betterHeuristics} é admissível, já que o custo real de
se fazer a rotação é no máximo de custo \texttt{4*Problem.StepCost()}, enquanto que na heurística
a estimativa é de no máximo 4 com relação a rotação.
} 

\end{document}
